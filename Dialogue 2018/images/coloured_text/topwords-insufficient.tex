\documentclass[a4paper,12pt]{article}
\usepackage{"D:/Document\space &\space Files/Documents/Dropbox/Study/tools/latex/mystyle"}
\graphicspath{ {images/} }
% \DeclareMathOperator{\tg}{tg} = \newcommand{\tg}{\mathop{\mathrm{tg}}\nolimits}
% \includeonly{task02}

\definecolor{my-red-base}{RGB}{193, 0, 32}

\colorlet{my-red}{my-red-base!80!}
\colorlet{my-red-light}{my-red-base!25!}

\author{Алексеев Василий, 474}
\title{}
\date{}
\begin{document}
%\maketitle
  %\thispagestyle{empty}
  %\newpage
  \pagenumbering{arabic}
%\tableofcontents
  %\newpage
%\addcontentsline{toc}{section}{Пролог}
  %\include{prologue}

  \noi
  Напротив, если предположить \mycolorbox{my-red-light}{существование} \mycolorbox{my-red-light}{суперсимметрии}, то введение новых \mycolorbox{my-red}{\textbf{частиц}} приводит как раз к такому объединению.
  \mycolorbox{my-red-light}{Оказывается}, что \mycolorbox{my-red-light}{суперсимметрия} не только обеспечивает объединение \mycolorbox{my-red-light}{взаимодействий}, но и стабилизирует объединённую \mycolorbox{my-red-light}{теорию}, в которой присутствуют два совершенно разных масштаба: масштаб \mycolorbox{my-red-light}{масс} \mycolorbox{my-red-light}{обычных} \mycolorbox{my-red}{\textbf{частиц}} (порядка $100$ \mycolorbox{my-red-light}{масс} \mycolorbox{my-red-light}{протона}) и масштаб великого объединения (порядка $10^{16}$ \mycolorbox{my-red-light}{масс} \mycolorbox{my-red-light}{протона}).
  Последний масштаб уже близок к так называемому \mycolorbox{my-red-light}{планковскому} масштабу, равному \mycolorbox{my-red-light}{обратной} \mycolorbox{my-red-light}{ньютоновской} \mycolorbox{my-red-light}{константе} тяготения, что \mycolorbox{my-red-light}{составляет} порядка $10^{19}$ \mycolorbox{my-red-light}{масс} \mycolorbox{my-red-light}{протона}.
  На этом масштабе мы \mycolorbox{my-red-light}{ожидаем} проявление \mycolorbox{my-red-light}{эффектов} \mycolorbox{my-red-light}{квантовой} \mycolorbox{my-red-light}{гравитации}.
  В этом моменте нас \mycolorbox{my-red-light}{ожидает} приятный \mycolorbox{my-red-light}{сюрприз}.
  Дело в том, что \mycolorbox{my-red-light}{гравитация} всегда стояла несколько особняком по отношению к остальным \mycolorbox{my-red-light}{взаимодействиям}.
  \mycolorbox{my-red-light}{Переносчик} \mycolorbox{my-red-light}{гравитации}, \mycolorbox{my-red-light}{гравитон}, имеет \mycolorbox{my-red-light}{спин} $2$, в то время как \mycolorbox{my-red-light}{переносчики} остальных \mycolorbox{my-red-light}{взаимодействий} имеют \mycolorbox{my-red-light}{спин} $1$.
  Однако \mycolorbox{my-red-light}{суперсимметрия} перемешивает \mycolorbox{my-red-light}{спины}.

  \bigskip

  \noi
   first \mycolorbox{my-red-light}{top words} of \emph{topic 3: физика} with \mycolorbox{my-red}{top 10} in bold: \textbf{частица}, \textbf{электрон}, \textbf{кварк}, \textbf{атом}, \textbf{энергия}, \textbf{вселенная}, \textbf{фотон}, \textbf{физика}, \textbf{физик}, \textbf{эксперимент}, масса, теория, свет, симметрия, протон, эйнштейн, нейтрино, вещество, квантовый, ускоритель, детектор, волна, эффект, свойство, спин, гравитация, материя, адрон, поль, частота

\end{document}