\documentclass[a4paper,12pt]{article}
\usepackage{"D:/Document\space &\space Files/Documents/Dropbox/Study/tools/latex/mystyle"}
\usepackage{"D:/Document\space &\space Files/Documents/Dropbox/Study/tools/latex/mystylelocal"}
\graphicspath{ {images/} }
% \DeclareMathOperator{\tg}{tg} = \newcommand{\tg}{\mathop{\mathrm{tg}}\nolimits}
% \includeonly{task02}

\definecolor{my-red-base}{RGB}{193, 0, 32}

\colorlet{my-red}{my-red-base!80!}
\colorlet{my-red-light}{my-red-base!25!}

\author{Алексеев Василий, 474}
\title{}
\date{}
\begin{document}
%\maketitle
  %\thispagestyle{empty}
  %\newpage
  \pagenumbering{arabic}
%\tableofcontents
  %\newpage
%\addcontentsline{toc}{section}{Пролог}
  %\include{prologue}

  \noi
  Напротив, если предположить \mycolorbox{my-red-light}{существование} \mycolorbox{my-red-light}{суперсимметрии}, то введение новых \mycolorbox{my-red}{\textbf{частиц}} приводит как раз к такому объединению.
  \mycolorbox{my-red-light}{Оказывается}, что \mycolorbox{my-red-light}{суперсимметрия} не только обеспечивает объединение \mycolorbox{my-red-light}{взаимодействий}, но и стабилизирует объединённую \mycolorbox{my-red-light}{теорию}, в которой присутствуют два совершенно разных масштаба: масштаб \mycolorbox{my-red-light}{масс} \mycolorbox{my-red-light}{обычных} \mycolorbox{my-red}{\textbf{частиц}} (порядка $100$ \mycolorbox{my-red-light}{масс} \mycolorbox{my-red-light}{протона}) и масштаб великого объединения (порядка $10^{16}$ \mycolorbox{my-red-light}{масс} \mycolorbox{my-red-light}{протона}).
  Последний масштаб уже близок к так называемому \mycolorbox{my-red-light}{планковскому} масштабу, равному \mycolorbox{my-red-light}{обратной} \mycolorbox{my-red-light}{ньютоновской} \mycolorbox{my-red-light}{константе} тяготения, что \mycolorbox{my-red-light}{составляет} порядка $10^{19}$ \mycolorbox{my-red-light}{масс} \mycolorbox{my-red-light}{протона}.
  На этом масштабе мы \mycolorbox{my-red-light}{ожидаем} проявление \mycolorbox{my-red-light}{эффектов} \mycolorbox{my-red-light}{квантовой} \mycolorbox{my-red-light}{гравитации}.
  В этом моменте нас \mycolorbox{my-red-light}{ожидает} приятный \mycolorbox{my-red-light}{сюрприз}.
  Дело в том, что \mycolorbox{my-red-light}{гравитация} всегда стояла несколько особняком по отношению к остальным \mycolorbox{my-red-light}{взаимодействиям}.
  \mycolorbox{my-red-light}{Переносчик} \mycolorbox{my-red-light}{гравитации}, \mycolorbox{my-red-light}{гравитон}, имеет \mycolorbox{my-red-light}{спин} $2$, в то время как \mycolorbox{my-red-light}{переносчики} остальных \mycolorbox{my-red-light}{взаимодействий} имеют \mycolorbox{my-red-light}{спин} $1$.
  Однако \mycolorbox{my-red-light}{суперсимметрия} перемешивает \mycolorbox{my-red-light}{спины}.

  \vspace{0.5cm}

  \noi
  Первые \mycolorbox{my-red-light}{топ-слова} темы \emph{<<физика>>}, включая первые \mycolorbox{my-red}{\textbf{топ-10-слов}} с вероятностями (\%):\\[0.2cm]
  \textbf{частица} (\emph{$2.7$}),
  \textbf{электрон} (\emph{$1.5$}),
  \textbf{кварк} (\emph{$1.5$}),
  \textbf{атом} (\emph{$1.3$}),
  \textbf{энергия} (\emph{$1.2$}),
  \textbf{вселенная} (\emph{$1.1$}),
  \textbf{фотон} (\emph{$1.0$}),
  \textbf{физика} (\emph{$0.9$}),
  \textbf{физик} (\emph{$0.9$}),
  \textbf{эксперимент} (\emph{$0.9$}),
  масса (\emph{$0.7$}),
  теория (\emph{$0.7$}),
  свет (\emph{$0.7$}),
  симметрия (\emph{$0.7$}),
  протон (\emph{$0.7$}),
  эйнштейн (\emph{$0.5$}),
  нейтрино (\emph{$0.5$}),
  вещество (\emph{$0.5$}),
  квантовый (\emph{$0.5$}),
  ускоритель (\emph{$0.5$}),
  детектор (\emph{$0.4$}),
  волна (\emph{$0.4$}),
  эффект (\emph{$0.4$}),
  свойство (\emph{$0.4$}),
  спин (\emph{$0.4$}),
  гравитация (\emph{$0.4$}),
  материя (\emph{$0.4$}),
  адрон (\emph{$0.4$}),
  поль (\emph{$0.4$}),
  частота (\emph{$0.4$})

\end{document}